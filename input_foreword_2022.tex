%\documentclass{sf2a-conf2022}
%\begin{document}

\TitreGlobal{SF2A 2022}

\title{Foreword}

\runningtitle{Foreword}

\maketitle


\author{E. Lagadec, President of the SF2A}

\vskip 25truemm

The French Society of Astronomy and Astrophysics ({\it Soci\'et\'e Fran\c{c}aise d'Astronomie et d'Astrophysique} -
SF2A) holds its annual meeting every spring. It is a great opportunity for all members of the french community to gather, exchange about different projects and work together on making our community stronger.
The 2022 edition was special, as it was the first in-person meeting following the COVID\,19 crisis.
 The 2020 meeting was held online and limited to the general assembly and a career session, always important for the non-permanent members of our community and the 2021 edition was a full meeting, but online.
 For two years, people could not meet in person and, as a community, we all felt the need to finally see each other again. It was particularly important for the youngest members of the community, as for some PhD student, the 2022 meeting was the first "real" meeting.



The meeting was held in the beautiful city of Besançon, and I want to thank the Local Organising committee for the organisation, with a  particularly beautiful conference diner on the historical site of the observatory. The Local Organizing Committee was chaired by Jean-Marc Petit and composed of Sarah Anderson, Anne Boetsch, Azzedine Lakhlifi, Olivier Lépine, Julien Montillaud, Benoît Noyelle, Emilie Piroutet, Florence Pretot, Thomas Ravinet and Philippe Rousselot.
I would like to thank them and all the members of the SF2A board who were all very active for preparing the meeting:
Nadège Lagarde, Annie Robin, Julien Malzac, Arnaud Siebert, Olivia Venot, Mamadou N’Diaye, Danielle Briot, Patrick de Laverny, Jean-Baptiste Marquette, Frédéric Pitout, Johan Richard.


%
The call for contributions to organise parallel sessions received a lot of answers, which allowed us to propose a very rich program, with meetings of the different \emph{Programmes Nationaux} and \emph{Actions Sp\'ecifiques} of INSU-CNRS and scientific meetings proposed by community members. We also received some very interesting ``societal" proposals, such as education for astronomy, collaborations between pro and amateur astronomers, the environmental transition, well-being of students and post-docs, and inequalities between men and women.

The meeting, called  ``\emph{Semaine de l'astrophysique fran\c{c}aise}'', or ``\emph{Journ\'ees de la SF2A}'' (and just ``SF2A'' by many colleagues)  included
the general assembly of our society, plenary sessions aimed at a large audience of professionals, and 20 parallel sessions  dedicated to various scientific and community issues and also the young researcher and thesis prize talks.  It was a great success, with more than 350 registered participants, and a great attendance for all the meetings, especially the societal ones. It confirmed us that this in-person moment was much needed, and that societal issues are very important to our community. SF2A meetings offer a unique place for such discussions to occur, and we hope it will continue in the future and that it will be important for the well-being of our community as a whole.


During the plenary sessions, re\-pre\-sen\-ta\-ti\-ves of INSU (M. Giard), CNES (P. Laudet) and of the ``\emph{sections}'' of CNAP (D. Mourard), CNRS (B. Famaey) and CNU (L. Rezeau) presented an update on the current situation of astrophysics in France. The discussion about careers is always important for students and post-docs, even if the small number of positions can be depressing.

The different \emph{Programmes Nationaux} and \emph{Actions Sp\'ecifiques} of INSU-CNRS proposed excellent speakers presenting the latest results from e.g. MATISSE at the VLTI, Solar orbiter, gravitational waves, galaxies with the JWST and LOFAR. Presentations of the CNES activities were also presented, together with the new \emph{Comité exoplanètes Transverse}, the \emph{Unité mixte internationale Chili} and a special presentation for the 50 years of CDS. The SF2A council also  decided to have discussions about two very important topics during the plenary sessions:
\begin{itemize}
\item Carbon footprint of astronomical infrastructures via a presentation by Jürgen Knödlseder 
\item Men-women inequalities in astronomy, with a presentation in plenary to introduce the parallel session.
\end{itemize}

We also invited Marc Audard, who came as a neighbour to present us the invited society: the Swiss Astronomical Society. He also gave us the latest news about the European Astronomical Society.

Afternoons were dedicated to parallel workshops covering all branch of astronomy. These 20 workshops were selected among propositions from the \emph{Programmes Nationaux} and \emph{Actions Sp\'ecifiques}, but also from individual members of the society. These workshops thus covered the interests of our whole community, in good accordance with the topicality of the field.

During these days, we  had the opportunity to celebrate the laureates of the 2002 SF2A prizes, so we want to warmly congratulate:

\begin{itemize}
    \item Audrey Coutens, \emph{Prix Jeune Chercheur 2022} laureate
    \item Benjamin Crinquand, \emph{Prix de Thèse 2022} laureate
    \item Quentin A. Parker, Pascal Le D\^u  an their team for the project \emph{Search for and Confirmation of Planetary Nebulae Candidates}, \emph{Prix Gémini 2022 de la collaboration pro-amateur} laureate
    \item Sarah Joiret, \emph{Prix Camille Flammarion 2022 de la médiation scientifique} laureate
\end{itemize}
Every year, the \emph{Découvrir l'Univers} Prize is awarded during the SF2A. It was awarded to the class of \emph{CE2-CM1 de l'école de Chalezeule}  for a very captivating and didactic work to explain the trajectory of the sun and the notion of relative movement. Congratulations to everyone!

Finally, we held our general assembly where the moral and financial reports were presented. A vote from the members will complement these presentations by the end of 2022.


The SF2A council wants to warmly thanks CNRS-INSU, CEA, CNES, UTINAM, Ville de Besançon and Université Bourgogne Franche-Comté for their financial support, which made the organisation of the meeting possible. 

%
Next year we will meet in Strasbourg, so see you in 2022 in beautiful Alsace!


\begin{center}
Eric Lagadec, 
President of the SF2A 
\end{center}


%\end{document}




