

%\documentclass{sf2a-conf2021}
%\begin{document}

\TitreGlobal{SF2A 2021}

%\title{Foreword}

\runningtitle{Foreword}

\setcounter{page}{1}
\maketitle


\section*{\Large \bf  Foreword}
\author{E. Lagadec, President of the SF2A}

\vskip 25truemm

The French Society of Astronomy and Astrophysics ({\it Soci\'et\'e Fran\c{c}aise d'Astronomie et d'Astrophysique} -
SF2A)  usually holds its annual meeting every spring. It is a great opportunity for all members of the French community to gather,
exchange about different projects and work together on making our community stronger.
Unfortunately, we have all gone through difficult times in 2020 and 2021 due to the sanitary crisis. The 2020 meeting was held online and limited to the general assembly and a career session, always important for the non-permanent members of our community.
For 2021, the sanitary situation was still complicated, but together with the SF2A council, we decided it was very important to hold a meeting, and we decided to go online. The 2020 and 2021 meetings were scheduled in Paris, and the Local Organising Committee, chaired by Paola Di Matteo had already done a great job at preparing it, but the uncertainty was too important to prepare an in-person meeting. We wish to warmly thank Paoloa Di Matteo and the 2020 and 2021 LOCs for the great work achieved to end up having no meeting in Paris. Thank you Paola Di Matteo, Misha Haywood, Kevin Baillié, Danielle Briot, Sylvain Cnudde, Florence Durret, Guillaume Hébrard, Frédéric Royer, Zakaria Meliani, Jaqueline Plancy and Stéphane Thomas.


The meeting was thus held online, and we were soon to face an important challenge: organise an online meeting and make sure the community enjoys it after more than a year of  online meetings. On the bright side, having an online meeting made it simple to host more parallel sessions than during in-person meeting, where the main limitation is the number of available rooms. But that meant we had to find a platform enabling exchanges, to record meetings etc.. while keeping the registration fees as low as possible. Quentin Kral offered us to be beta-testers for his CarbonFreeConf platform. Carbonfree is  designed to organise online meetings, which should be more and more frequent with the climate crisis we are facing. We thus felt it was a good opportunity to use this service, offered at its cost price by Quentin Kral. I want to warmly thank him for that and his amazing reactivity before, during and after the meeting. It was quite some work for us, but at the end it worked well, and we hope it will be useful for online meetings in the future.
%
The call for contributions to organise parallel sessions received a lot of answers, which allowed us to propose a very rich program, with meetings of the different \emph{Programmes Nationaux} and \emph{Actions Sp\'ecifiques} of INSU-CNRS and scientific meetings proposed by community members. We also received some very interesting "societal" proposals, such as the future of French historical observatories, discussions between scientists and journalists, sociological studies of observatories, well-being of students and post-docs, and inequalities between men and women.

The meeting, called  ``\emph{Semaine de l'astrophysique fran\c{c}aise}'', or ``\emph{Journ\'ees de la SF2A}'' (and just ``SF2A'' by many colleagues)  included
the general assembly of our society, plenary sessions aimed at a large audience of professionals, and 27 parallel sessions  dedicated to various scientific and community issues and also the young researcher and thesis prize talks.  It was a great success, with more than 800 registered participants, and a great attendance for all the meetings, especially the societal ones. It confirmed us that the 2021 SF2A meeting was much needed, and that societal issues are very important to our community. SF2A meetings offer a unique place for such discussions to occur, and we hope it will continue in the future and that it will be important for the well-being of our community as a whole.



%
During the plenary sessions, re\-pre\-sen\-ta\-ti\-ves of INSU (G. Perrin), CNES (P. Laudet) and of the ``\emph{sections}'' of CNAP (D. Mourard), CNRS (B. Mosser) and CNU (L. Rezeau) presented an update on the current situation of astrophysics in France. The discussion about careers is always important for students and post-docs, even if the small number of positions can be depressing.

The last news from large observatories were given by  B. Devost for CFHT and  M. Cirasuelo for ESO. The different \emph{Programmes Nationaux} and \emph{Actions Sp\'ecifiques} of INSU-CNRS proposed excellent speakers presenting the latest results from Gravity/MATISSE at the VLTI, Insight on Mars, tensions about the Cosmic Microwave Background, sample returns from asteroids, the potential of pulsar timing networks for the detection of gravitational waves, the Parker Solar Probe, open access to scientific data and an update on the study of molecules in molecular clouds. The SF2A council also proposed some talks about "hot topics" such as the Great Dimming of Betelgeuse, news about the Perseverance Rover on Mars and Astronomy in Africa. We also decided to have discussions about three very important topics during the plenary sessions:
\begin{itemize}
\item The preservation of our night sky via a presentation by Piero Benvenuti of the IAU action to preserve dark and quiet skies for astronomy and humanity.
\item Well-being for PhD students and post-docs by Natalie Webb, an important topic for the most precarious in our community, even more after almost two year of pandemics.
\item Men-women inequalities in astronomy, with a full session in plenary to introduce the parallel session. This is a very important topic for SF2A, and we have formed a commission to work on it. The talks were presented by Isabelle Kraus, specialist of the topic and VP of the Strasbourg University. Guy Perrin presented the INSU situation and actions that will be conducted.
\end{itemize}

Because of the pandemics, we could not organise in-person ceremonies for the SF2A prize laureates. They were thus invited to give talks during the plenary sessions, so we want to warmly thank:

\begin{itemize}
    \item Anaëlle Maury, \emph{Prix Jeune Chercheur 2020} laureate
    \item Doogesh Kodi Ramanah, \emph{Prix de Thèse 2020} laureate
    \item Alexandre Santerne an his team for the project \emph{Detection and Follow-Up of Exoplanets by Amateur Astronomers}, \emph{Prix Gémini 2020 de la collaboration pro-amateur} laureate
    \item Matthieu Béthermin, \emph{Prix Jeune Chercheur 2021} laureate
    \item Lisa Bugnet, \emph{Prix de Thèse 2021} laureate 
    \item Marc Delcroix and  Ricardo Hueso for the project \emph{Collaborative Amateur-Professional project for the detection and characterisation of impacts on Jupiter}, \emph{Prix Gémini 2021 de la collaboration pro-amateur} laureate
\end{itemize}

Finally, we held our general assembly where the moral and financial reports were presented. A vote from the members will complement these presentations by the end of 2021.


Afternoons were dedicated to parallel workshops covering all branch of astronomy. These 27 workshops were selected among propositions from the \emph{Programmes Nationaux} and \emph{Actions Sp\'ecifiques}, but also from individual members of the society. These workshops thus covered the interests of our whole community, in good accordance with the topicality of the field.


Every year, the \emph{Découvrir l'Univers} Prize is awarded during the SF2A. It was awarded to the class of Mireille Jarlut at the\emph{Ecole Maternelle Thérèse Roméo} in Nice for a very touching work to explain the day/night alternation and gravity to very young pupils. Congratulations to everyone!

The SF2A council wants to warmly thanks CNRS-INSU for its financial support, which made the organisation of the meeting possible. 

I would like to thank all the members of the SF2A board who were all very active for preparing the meeting, with a special and warm thank to all the always happy and fast-answering members of the \emph{SAV}:

Kevin Baillié, Quentin Kral, Nadège Lagarde, Julien Malzac, Jean-Baptiste Marquette, Mamadou N'Diaye,  Johan Richard, Arnaud Siebert, Olivia Venot.
They all made this meeting a success. 
%
Next year we should finally meet in person, and I think many people are excited about it.. so see you in 2022 in Besançon!


\begin{center}
Eric Lagadec, 
President of the SF2A 
\end{center}


%\end{document}
