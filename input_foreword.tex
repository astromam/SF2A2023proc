%\documentclass{sf2a-conf2023}
%\begin{document}

\TitreGlobal{SF2A 2023}

\title{Foreword}


%\setcounter{page}{1}
\maketitle


\author{Nad\`ege Lagarde, President of the SF2A}

\vskip 15truemm

As the new president of the French Society of Astronomy and Astrophysics (Société Fran\c{c}aise d’Astronomie et d’Astrophysique - SF2A), it is an honour for me to write this foreword. As emphasized by my predecessors, the annual meeting of the SF2A is a highly significant event for our community, a place where permanent, doctoral and post-doctoral researchers can exchange their ideas and present to the whole community their new discoveries. The 2023 edition took place from June 20th to June 23rd in Strasbourg, having been efficiently prepared by members of the \textit{Observatoire Astronomique de Strasbourg}, of the \textit{Centre de données astronomiques de Strasbourg} and of the \textit{Jardin des sciences de l'Université de Strasbourg}. \\

As every year, this meeting included the general assembly of our society, plenary sessions aimed at a large audience of professionals, workshops dedicated to various scientific issues, the young researcher and thesis price ceremony, events for schools and the public, discussions concerning societal subjects. This 2023 edition was a great success, with a record number of participants (more than 450) and marked a turning point in the way the conference was organised. 
For the first time, all the plenary lectures and workshops were accessible online. This enabled researchers who were unable to attend SF2A meeting  in person to take an active part in the conference, sharing their work and exchanging views with other participants. This innovative approach is particularly important in the current context, where preserving the environment has become an absolute priority for everyone. I would like to thank the local organising committee for their exceptional efforts and remarkable efficiency in implementing this new format.\\

The programme of the plenary sessions was, as usual, very rich, with several high-quality presentations proposed by the various Programmes Nationaux and Actions Spécifiques of INSU-CNRS, covering a broad range of astrophysical topics, including news from Gaia and JWST, planetology, asteroseismology. Representatives of institutions (M. Giard for INSU, T. Bret-Dibat  for CNES), the CNU (S. Mei), and the “sections” of CNRS (B. Famaey) and CNAP (D. Mourard) presented us a detailed view of the french research situation with crucial informations for the career. This whole session was concluded by a very interesting discussion with the community.\\

The place of women in the science community is a very important topics, specially in physics and astronomy and the SF2A meeting was an excellent opportunity to discuss it.
R.-M. Ouazzani, who is in charge of \textit{the SF2A's Women and Astronomy Committee}, gave us a presentation of the committee's latest activities. It is also very important to think about environmental issues. In this context, \textit{the SF2A's Environmental Transition Committee} invited E. Guilyardi to give a presentation on integrating environmental issues into the conduct of research. The SF2A Board also wanted to highlight an important societal issue by inviting J.-G. Cuby to give a presentation on \textit{``Astronomy on indigenous lands: the case of Maunakea"}. These three presentations attracted a great deal of interest from the community, as the very interesting discussions that followed showed. \\

The call for contributions to organise parallel sessions received a lot of answers, which allowed us to propose a very rich program, with meetings of the different \emph{Programmes Nationaux} and \emph{Actions Sp\'ecifiques} of INSU-CNRS and scientific meetings proposed by community members. Indeed, this year, not less than eighteen scientific workshops were organized on topical subjects, and three additional sessions were devoted to \textit{``Gender equality in A\&A: gender stereotypes in teaching"} ; \textit{``Environmental transition: what levers for action to reduce the environmental footprint of astronomy?"} and to \textit{``Well-being in astrophysics"}, each of which brought in specialists in these societal issues to enhance our thinking on these key subjects. This “record” number of sessions obviously reflects the great wealth and breadth of exchanges across thematic boundaries as well as the growing role played by this meeting and SF2A in animating our community and in initiating fruitful discussions and collaborations in a friendly and stimulating atmosphere.
Finally, the SF2A annual meeting were concluded by our general assembly where the moral and financial reports were presented by the president and treasurer of the SF2A Council, N. Lagarde and K. Baillie. A vote from the members has complemented these presentations in Autumn 2023.\\

Three special moments usually illuminated the SF2A meeting. The first one was the ”outreach” conference by Roland Le Houc entitled \textit{"Talking science through fiction"}. The second one was the SF2A prize ceremony. The laureate of the Thesis prize was \textbf{Marta de Simone} who presented her brilliant work concerning Hot corinos and the early organic molecular enrichment of the planet formation zones. The laureate of the young researcher price was \textbf{Laure Ciesla} for her work on the star formation history of galaxies based on their spectral emission. The laureate of the Camille Flammarion for scientific mediation (SF2A-SAF) was \textbf{Daniela Galarraga-Espinosa} for her participation in the \textit{``Women in Sciences in Equador"} project. This ceremony was followed by a projection and a dinner-cocktail in the new and impressive planetarium of Strasbourg. Last but not least, an important moment of the week was the School project prize called ``Découvrir l’Univers”. I would like to highlight the incredible collaboration between the LOC and \textit{Jardin des sciences de l'Université de Strasbourg}, who did an outstanding work in organising this prize. The jury was really impressed by the work of all the classes taking part in the competition. The highlight of the competition took place during the days when the competition participants were invited to a lecture given by Roland Le Houc. \\

I wish to thank the INSU-CNRS, the Observatoire de l'Université de Strasbourg, the ``Centre de données astronomiques de Strasbourg" as well as the CNES, the CEA, the National Programs and Actions Spécifique of INSU-CNRS, for their financial and organisational support. We are extremely grateful to the University of Strasbourg for hosting our meeting. The 2023 edition was held in the beautiful city of Strasbourg, and I want to thank the Local Organising committee for the organisation, with a particularly beautiful conference diner on the historical site of the observatory. The Local Organizing Committee was chaired by Arnaud Siebert and composed of Mark Allen, Thibault Barnouin, Matthieu Bethermin, Paolo Bianchini,  Lucie Correia, Pierre-Alain Duc, Jonathan Freundlich, Céline Halter, Yassin Khalil Rany, Frédéric Marin, Nicolas Martin, Mei Palanque, Emeline Ricciuti (JDS), Clément Stahl, Srikanth Togere Nagesh,  Mathias Urbano,  Karina Voggel and Milène Wendling (JDS). I would like also to thank all the members of the SF2A board who were all very active for preparing the meeting : Olivia Venot, Kévin Baillié, Arnaud Siebert, Johan Richard, Rhita-Maria Ouazzani, Julien Malzac, Mamadou N'Diaye, Eric Lagadec, Danielle Briot, Frédéric Pitout and Matthieu Berthermin.  
Together with the LOC members, they made the meeting a success. In addition to the rich program mentioned above (and somewhat detailed in these proceedings), embracing all of the astronomy research performed in France, the work of the LOC has also offered to all the participants many opportunities to meet and discuss beyond the limits of their own field of expertise.\\

See you next year in Marseille!


\begin{center}
Nadège Lagarde, 
President of the SF2A 
\end{center}


%\end{document}
